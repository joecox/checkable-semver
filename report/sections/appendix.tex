\section{Appendix}
This appendix describes how to run our tools to reproduce our
results. Please note that reproducing all the results is extremely CPU
intensive and will take many hours.

To download the code, clone the repository at
\begin{center}
  {\tt https://github.com/joecox/checkable-semver}
\end{center}

You will also need to clone Mocha at
\begin{center}
  {\tt https://github.com/mochajs/mocha}
\end{center}

We explictly recommend against running these tools on CentOS. Doing so
results in many unexplainable errors.

\subsection{Running {\large {\tt detect}}}

The {\tt detect} tool takes a number of options, including a version
tag, the name of the test directory, the path to the repository you
are testing, and the test suite to run. To find violations on version
{\tt \$v}, use the command

\begin{center}
  {\tt ./detect \$v -d test -r <mocha\_dir> -s jsapi}
\end{center}
where {\tt \$v} is a variable assigned to a version tag, such as {\tt
  1.2.0}. This will run the relevant Add Feature tests and Breaking
Change tests and report any violations.

If you wish to generate all violations (which provides the data for
figures \ref{fig:violations} and \ref{fig:cumulative}), run the
commands

\begin{center}
  {\tt ./setup-all-the-tests.sh}
\end{center}
\begin{center}
  {\tt ./run-all-the-tests.sh}
\end{center}
Please note that this will take several hours and is extremely CPU
and disk-intensive.

\subsection{Running {\large {\tt simulate}}}

To run the {\tt simulate} tool, you must pass the repository path.

\begin{center}
  {\tt ./simulate -r <mocha\_dir>}
\end{center}
This will print out lines showing the actual tags and simulated tags,
side by side. This raw data is used to produce figure \ref{fig:simulation}.
