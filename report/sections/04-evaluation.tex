\section{Evaluation}
In this section, we describe how we will evaluate the effectiveness
and practicality of our discipline.

\subsection{Quantitative Evaluation}
Our first evaluation method is a quantitative analysis of running our
tool on an existing repostory. In doing so we hope to answer:

{\bf RQ1.} Does our tool detect SemVer violations?

We know from our proof sketches in section \ref{sec:cvt} that if a
library interface is fully specified, we can accurately detect SemVer
violations. This statement is subject to multiple confounding
factors, including whether tests are expressive enough to detect
violations and whether tests are too strict, which would result in an
over-approximation of the interface.

We evaluate our tool and the cross-version testing that underlies it
by using Mocha's {\tt jsapi} test suite as an interface specification
to find SemVer violations.
%
We run our {\tt detect} tool on each release in the library and
collect the Breaking Change and Added Feature violations for each
release. For each version, we report the number violated tests. A test
may be duplicated in the results if it appears is in multiple versions
but this may emphasize the importance of the test, since it survived
for multiple releases.

The are several confounding factors to this evaluation. First is the
fact that these tests do not necessarily conform to our
discipline. However, this does not affect the performance of the tool,
only our ability to draw meaningful conclusions about potentional
SemVer violations from its results. Still, our violation density
analysis provides some justification in using {\tt jsapi} as an
interface spec.

Another confounding factor is related to the problem of test
coverage. Tests which are too specific may not fully cover all
features of a program. It is therefore reasonable to ask if an
incomplete specification still can report meaningful errors.  We
evaluate our tool using the Mocha test framework~\cite{mocha} as our
subject library. Mocha is an ideal subject because it has many
versions and is well tested.

A third confounding factor is that too many violations may be
reported, which reduces the practicality of our tool. This confounding
factor is also the most controversial. Many developers have claimed
that adhering to SemVer, given their internal definition of an
interface, would force them to bump the major version on every
update~\cite{backbone-2888,exoplayer-1382,crawford-not-semver}. This
problem would be even more pronounced if an interface was clearly
specified using tests and SemVer compliance was checked with a strict
and uncompromising tool. To address this factor, we have created a
simulated version history of Mocha using our approach. This version
history is evidence that following our discipline would not be an
undue burden on library authors. It is also a worst-case scenario,
where developers did not expect their tests to be used to describe a
public interface and were not warned about violations before
releasing.

\subsection{User Study Evaluation}
Our second evaluation moethod is a user study of the efficacy of our
approach. We attempt to answer:

{\bf RQ2.} Can our tool detect SemVer violations without a interface
test suite that adheres to our discipline?

{\bf RQ3.} How difficult is it to adopt our discipline of writing interface
tests?

To answer {\bf RQ2}, we manually inspected the {\em potential}
violations reported by our tool in order to determine whether any of
them represent {\em actual} violations. In particular, we determine
whether each test failure was due to an illegal change in the
interface (in which case the failure is a true violation), due to
non-adherence to our discipline, or for some other reason.

While reviewing the violations for {\bf RQ2}, we refactored failing
tests to adhere to our discipline whenever possible. To answer {\bf
  RQ3}, we recorded the time required for the refactoring. We also
looked for any inherent limitations of our approach that may make
adherence more difficult.

This study was conducted by the authors, who had no prior knowledge of
the internals of Mocha. Lacking ground truth for what the ``true''
interface is, we consulted the Mocha documentation, source code and
comments, and version history to make the best determination
possible. Whenever multiple interpretations were possible, we gave
Mocha developers the benefit of the doubt, choosing the interpretation
that would minimize SemVer violations.

