\section{Motivation}

It has always been difficult to rigorously specify library interfaces.
Interfaces are often defined with text, which is often incomplete and not
machine checkable,  or,  for some programming languages, with types, which are
often incomplete. Machine checkability is important as it prevents
documentation to go stale. Completeness is important if the API is the only
means of communication between the library and the client. And this is only for
one version of the API. When looking at the evolution of an API. It is harder
still to ensure that the API does not introduce a breaking change unexpectedly.
Semantic Version, or SemVer, helps library authors define the scope of their
changes solely through the version number.  However, authors can and often do
violate SemVer. For example, an author may make a backwards incompatible change
but only increment the minor version.  Library consumers observing the version
number will not expect the change and systems will break.

In this paper we define the scope of an interface test, and describe a
discipline for how to write and maintain the tests. We have designed a tool
that using these tests we will detect violations to SemVer, which will allow
library developers to give stronger guarantees to their clients. 

Many libraries have unit tests, regression tests, integration tests, and more.
However, any given test may test the library interface, internal
implementation, or both.  The tests that test only the library interface are
interface tests.  In most libraries, they will be hiding among other tests.

We will create a tool that can identify SemVer violations given a code
repository, its testing framework, and a version history.  Our tool will apply
tests appearing in each version of a repository to later versions of the code.
Tests that succeed in an earlier version but fail in a later version are
candidates for SemVer violations.  We will manually identify tests that are
interface tests and the tool will further compare test behavior against
versioning.  For example, if an interface test succeeds in an earlier version
but fails in a subsequent minor update, then SemVer has been violated.  

Thus, a library author can use our tool to detect potential SemVer violations
by testing previous versions tests against the code the author is about to
commit. Our tool will detect any SemVer violations and give the author the
choice to either revert the API changes or push a major update.
