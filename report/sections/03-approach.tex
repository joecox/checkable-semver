\section{Approach}

Our approach consists of defining an interface specification, which is a
special kind of tests, and a new discipline for writing these tests, and a tool
that is able to detect SemVer violations in a library.

\subsection{Interface specifications}

\begin{definition}\textit{Interface Specification.}
An interface specification is a collection of tests that given an
implementation and returns a list of interface violations. 
$$\text{Spec} : \text{Impl} \to [\text{Violation}]$$
A violation is a identifier of a failed test. A specification
\begin{enumerate}
    \item does not test anything secret, e.i.\ something that is subject to change,
    \item and do test everything that a client can rely on.  
\end{enumerate}
\end{definition}

The interface is a two sided contract. The library, denotes clearly what can be
relied upon, but by using a library a client, commits to never rely on anything
that is not tested.  Our interface specification is both sound and complete by
definition.  Following the definition we have created a discipline of interface
specification using tests. We use the RFC 2119\footnote{The key words ``MUST'', ``MUST NOT'', ``REQUIRED'', ``SHALL'', ``SHALL NOT'', ``SHOULD'',
``SHOULD NOT'', ``RECOMMENDED'',  ``MAY'', and ``OPTIONAL'' in this document are to be
interpreted as described in RFC 2119\cite{rfc2119}.} standard to describe the
discipline.

\begin{enumerate}
    \item The library MUST clearly separate implementation specifications from
    other tests.
    \item Any released library version MUST be tested against the corresponding
    specification, and have no interface violations.
    \item Any feature exposed by the library, but not tested, SHOULD be
    considered a secret, and not be used by a client.
    \item A client MAY rely on any feature tested by the specification.
\end{enumerate}

This definition and discipline is flexible enough that can describe an
interface, while still being machine checkable. Any specification is a trade
between machine power and API expressed. This definition of cause has some
limitations. A test \texttt{assert add (2, 3) == 5}, only provides the existence proof,
that adding 2 and 3 equates 5. The client, can then violate our discipline and
assume that 2 and 3 are representative for all integers or be safe and only use
it with the arguments 2 and 3. Of cause different techniques as concolic
testing~\cite{sen05-cute} or random testing might improve the expressiveness of test. The
advantage of this approach is that the interface and its guarantees are
concretely described. 

\subsection{Semantic Versioning using Interface Specifications}

To show that the technique is useful in practice, we have chosen to implement
semantic versioning based of our interface specifications.

\begin{definition}\textit{Semantic Versioning} (SemVer) is a way of describing
intention, and giving guarantees using version numbers\cite{semver}.
Related to to the last version $X.Y.Z$, the next version has to
monotonically increasing, meaning that it has to satisfy one of the
following rules.

\begin{enumerate}
    \item A patch release has number $X.Y.Z’$, where $Z < Z’$. It may include
    bug fixes only, no new features or breaking changes.
    
    \item A minor release has number $X.Y’.Z’$, where $Y < Y’$. It may include
    bug fixes or new features, but no breaking changes.
    
    \item A major release has number $X’.Y’.Z’$, where $X < X’$. It may include
    bug fixes, new features, and breaking changes.
    
    \item Any releases in the range $0.Y’.Z’$, where $X = 0$ and $Y < Y’$ or $Z
    < Z’$ is unstable and the interface can change without warning.
\end{enumerate}

\end{definition}

We have chosen semantic versioning, because it is widely used in \emph{npm}, and
\emph{bower} to describe package dependencies. \marktodo{Both package managers depend 
heavily that dependencies follow the SemVer policy}{check this}. The problem is that
currently is semantic versioning, based solely on the intuition of the library
developers.  But developers can be wrong, and not assume that changes they have
made is violating the interface. We can give some guarantees, that the
implementation is following an interface which in turn is related to a semantic
version. 

Semantic versions talk about the relations between different releases of the
same library. So to verify a semantic version, we have to associate it with an
implementation and a specification. A release is  a tuple of a semantic
version, an implementation and a specification.
$$ \text{Release} : \text{Ver} \times \text{Impl} \times \text{Spec} $$

A release can only be validated in the context of the a release history. A
release can only be accepted if our tool finds no violations. The violations
can be described like this; given a new release with major version greater than
0 and a list of previous releases, the release has to pass the following
checks.

\begin{enumerate}
    \item The new release specs has to have no violations on the new
    implementation;
    \item the new release version must be greater than all previous versions, as
    described in the semantic versions;
    \item all previous implementations with versions in same minor version as the
    new release, must run on the new specification without violations;
    \item and all previous specifications in the same major version as the new
    release, must run the new implementation without violations.
\end{enumerate}

If all the steps are passed, the release is following SemVer. 

\begin{theorem}[Soundness]
Given a release and a release history, our algorithm will reject the release if
and if it violates SemVer in relation to the release history.
\end{theorem}

\begin{proof}
If a release is rejected, it must have failed in one of the 4 conditions. 
\begin{itemize}
  \item If the first condition failed, the release is invalid, and semantic
      version does not apply to broken releases. 
  \item If the second condition failed, then the patch did not follow the
      monotonic requirement of SemVer. 
  \item If the third requirement failed, some previous release, in the same
      minor version, have an implementation which violates something the specs
      of the current version. This means that the specs must be testing a new
      feature or have a breaking change, and this violates the first rule of
      SemVer.
  \item If the third requirement failed, some previous release, in the same
      major version,  must have a spec which is violated by the current
      implementation. This means that something which was guaranteed has been
      broken; which makes this a breaking change. Since the new release is at
      most a minor release compared, this violates the second rule of SemVer.
\end{itemize}
Because all possible failed cases is the result of a SemVer violation, the
algorithm is guaranteed to only reject a release if SemVer is violated.
\end{proof}

\begin{theorem}[Completness]
Given a release and a release history, our algorithm will accept the release if
and if adheed to SemVer in relation to the release history.
\end{theorem}

\begin{proof}
If a release is wrongly accepted it must have not satisfied all of the SemVer
rules. If the new release is a major release, there is no way to violate the
SemVer rules. If the new release is a minor release, it must contain a breaking
change to violate SemVer. This is impossible because 
\end{proof}
