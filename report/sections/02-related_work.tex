\section{Related Work}

\todo{Parnas paper} \cite{Parnas}.

A recent study from Carnegie Mellon University\cite{bogart15-break}, consists of seven interviews
with package maintainers of packages in CRAN (R) and Node.js (Javascript). A
joint problem seems to be that it is too complicated to follow breaking changes
in dependent libraries. Even though that Node.js has semantic versioning, a
Node.js developer claims the technique is not uniformly applied. The study
concludes that tools has to be created that are able to, more concisely, inform
developers about breaking changes, and let them reason about the stability of a
library.

Automatic semantic versioning has been efficiently used in Elm. Elm is
a functional and static typed language which makes semantic versioning
simple. In Elm the version numbers is defined like this: a patch is
never changing the API, a minor version added stuff to the API, and
major version changed or removed values in the API. API specifications
in Elm are implicitly defined by the types of all exposed values,
which includes both constants and functions, in all exposed
modules. This provides some nice guarantees, for example that a
dependency to a particular version will not introduce any type
errors. However, using types alone as API specifications has limited
expressiveness: any behavior not reflected by the types alone is
automatically considered an API secret. To cope with this limitation,
the library authors often define a more complete API specification in
documentation. The rules of Semantic Versioning should be followed
with respect to the complete API specification including this
documentation, but Elm's tool cannot enforce this. Users should not
blindly trust Elm to enforce semantic versioning without some
additional check of SemVer compliance. Our approach is one such
additional check, and is complimentary to Elm's approach.

In his seminal work on Delta Debugging, Zeller writes ``The GDB people have done
it again. The new release 4.17 of the GNU debugger brings several new features,
languages, and platforms, but for some reason, it no longer integrates properly
with my graphical front-end DDD: the arguments specified within DDD are not
passed to the debugged program. Something has changed within GDB such that it
no longer works for me. Something? Between the 4.16 and 4.17 releases, no less
than 178,000 lines have changed. How can I isolate the change that caused the
failure and make GDB work again?''

Delta Debugging aims to locate the change that caused the failure. If GDB is
following semantic versioning, there are two possible explanations; It's GDB's
fault: GDB published a minor version upgrade with a breaking change, violating
SEMVER; or It's DDD's fault: DDD unwittingly relied on (unspecified) secrets of
GDB. In either case, our tool would detect who was at fault and localize the
error. Further, it would help the GDB developers avoid releasing breaking
changes, and would help the DDD developers avoid relying on the secrets of GDB.

Change-impact analysis (CIA) \cite{Chianti,FaultTracer,Stoerzer} can
be used for regression-test selection or to help isolate the cause of
a regression test failure.
%
CIA can compliment our cross-version testing technique.
%
When our cross-version testing approach detects a SemVer violation,
the library authors will need to know which change(s) caused the
violation. In such cases, we can apply change-impact analysis
\cite{Stoerzer} to detect which changes from library version A induced
failures in the cross-version tests. Stoerzer et al. \cite{Stoerzer}
developed a change classification technique that classifies changes
that affect a failing test by their likelyhood to have induced the
failure. This would help developers focus their attention on the
changes most likely to have caused the SemVer
violation. Alternatively, it could help developers quickly factor out
a SemVer-compliant release within their deadline.


