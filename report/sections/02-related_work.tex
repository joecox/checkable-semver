\section{Related Work}

Parnas proposed \cite{Parnas} that systems should be decomposed into
modules that support parallel development. This runs counter to the
intuitive idea of decomposing the system using ``separation of
concerns'' as the only consideration (the so-called ``functional
decomposition''). The guiding principle developers should use in
modularizing their code is to separate the module interface, which
other modules can rely on behaving a particular way, from
implementation ``secrets'', which the module author has the freedom
the change. 

Semantic Versioning \cite{semver}, or SemVer, is complementary to the ideas
proposed by Parnas.
%
SemVer, fully defined in section \ref{sec:semver}, solves a number of
problems experienced by library authors and consumers, including how
to handle breaking changes.
%
SemVer allows library consumers to better control which interface
definitions they subscribe to. This is particularly important when
client code relies on libraries or modules developed by independent
teams with separate goals and timelines.
%
Semantic versioning strengthens Parnas' ideas by establishing a
contract between the authors and clients of a module, which governs
the evolution of the module. 
%
It allows clients to safely incorporate new upgrades to their
dependencies that may improve performance, fix bugs, or add new
features, without worrying about whether the upgrade will break their
code.
%
Our project aims to strengthen this even further, by supporting SemVer
with a testing discipline and tool support.

A recent study from Carnegie Mellon University \cite{bogart15-break}
consists of seven interviews with package maintainers of packages in
CRAN (R) and Node.js (Javascript). A common problem discovered by the
study is that breaking changes in package dependencies are too
complicated to follow and too hard to integrate into the authors'
packages. Even though Node.js has semantic versioning, a Node.js
developer claims that the technique is not uniformly applied across
Node packages. The study concludes that tools have to be created that
are able to more concisely inform developers about breaking changes
and let them reason about the stability of a library.

Automatic semantic versioning has been efficiently used in Elm. Elm
is a functional and statically typed language which makes semantic
versioning simple. In Elm, version number updates follow a pattern
similar to SemVer: a patch update changes nothing about the API, a
minor version update adds features to the API, and a major version
update changes or removes values in the API. API specifications in
Elm are implicitly defined by the types of all exposed values, which
include both constants and functions, in all exposed modules. This
provides some nice guarantees, such as the assurance that a
dependency to a particular version will not introduce any type
errors. However, using types alone as API specifications has limited
expressiveness: any behavior not reflected by the types alone is
automatically considered an API secret. To cope with this limitation,
library authors often define a more complete API specification in
documentation. The rules of SemVer should be followed with respect to
the complete API specification, including the portion found in
documentation, but Elm's tool cannot enforce this. Users should not
blindly trust Elm to enforce semantic versioning without an
additional check of SemVer compliance. Our approach is one such
additional check, and is complementary to Elm's approach.

In his seminal work on Delta Debugging, Zeller writes ``The new
release 4.17 of the GNU debugger brings several new features,
languages, and platforms, but for some reason, it no longer
integrates properly with my graphical front-end DDD...Something has
changed within GDB such that it no longer works for me. Something?
Between the 4.16 and 4.17 releases, no less than 178,000 lines have
changed. How can I isolate the change...''

Delta Debugging aims to locate the change that caused the failure. If
GDB is following semantic versioning, there are two possible
explanations. One explanation is that it's GDB's fault. GDB published
a minor version upgrade with a breaking change, violating
SemVer. Another explanation is that it's DDD's fault. DDD unwittingly
relied on (unspecified) secrets of GDB. In either case, our tool
would detect who was at fault and localize the error. Furthermore, it
would help the GDB developers avoid releasing breaking changes and
would help the DDD developers avoid relying on the secrets of GDB.

Change-impact analysis (CIA) \cite{Chianti,FaultTracer,Stoerzer} can
be used for regression-test selection or to help isolate the cause of
a regression test failure.
%
CIA can compliment our cross-version testing technique.
%
When our cross-version testing approach detects a SemVer violation,
the library authors will need to know which change(s) caused the
violation. In such cases, we can apply change-impact analysis
\cite{Stoerzer} to detect which changes from a specific library
version induced failures in the cross-version tests. Stoerzer et
al. \cite{Stoerzer} developed a change classification technique that
classifies changes that affect a failing test by their likelyhood to
have induced the failure. This would help developers focus their
attention on the changes most likely to have caused the SemVer
violation. Alternatively, it could help developers quickly factor out
a SemVer-compliant release within their deadline.


