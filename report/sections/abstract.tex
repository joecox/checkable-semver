\begin{abstract}
  Managing dependencies in systems composed of many
  in\-dep\-endent\-ly-developed modules can be tedious and error-prone.
%
  It can be unclear whether a dependency should be updated to a new
  version. On the one hand, the update could include important bug
  fixes, enhancements and new features. On the other hand, the update
  might also include subtle changes that introduce new bugs into the
  system.
%
  The current state of the practice is to use the convention of
  Semantic Versioning to communicate the possible differences between
  two releases of a module.
%
  However, the effectiveness of this is limited by the absence of
  clear module interface specifications.
%
  A client may mistakenly rely upon interface secrets, or a module author
  may accidentally use an incorrect version number for a new release.
%
  In either case, Semantic Versioning alone provides no help.

  We fill this gap by devising a new test-based discipline for
  defining module interface specifications, along with a {\em
    cross-version testing} approach that uses interface tests to
  detect Semantic Versioning.
%
  We realize our techniques in a tool for Node/JS libraries, and
  evaluate them using the Mocha testing framework as our subject
  library.
%
  Our results show promise for our techniques, but underscore that
  some investment is required from the module authors to adopt
  our discipline. 
\end{abstract}
