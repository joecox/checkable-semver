\section{Discussion}
One limitation of our approach is that refactoring assertions about
interface secrets out of interface tests can result in duplicated testing
code. This might be avoided by using a more complex test design, but
that could reducing the readability of test, making them less useful
as specifications.

We expected to find some interface tests that could not be effectively
refactored to remove interface secrets. We did not find any such tests, but
the small scope of our study prevents us from making any conclusions
about the likelyhood of this scenario in practice. 

Another limitation relates to our definition of interface
specification. As an example, a test \texttt{assert add (2, 3) == 5}
only checks that the sum of 2 and 3 is 5. The client may incorrectly
interpret this as saying that the sum of any two numbers is 5, or
even that the sum of any two objects is 5. We conjecture that such
problems could be addressed by adding more tests, or by using
techniques such as concolic testing~\cite{sen05-cute} or random
testing~\cite{quickcheck}. The advantage of our approach is that the
interface and its guarantees are concretely described and
machine-checkable.

\subsection{Threats to Validity}
An external threat to validity is that our tools only apply to the
Mocha testing framework. They do not function with other testing
frameworks. However, we believe our idea and approach is applicable
to any testing framework in any programming language, given some work
to port the tools.

A threat to internal validity is that the library we tested on was
itself a testing framework; as such it is possible that the test
suite is not representative of most library test suites. Therefore,
we might underestimate the difficulty of adopting our discipline.

A threat to conclusion validity is that we only tested our approach
on a small library. Also, it is possible that we did not test our
approach on a library with a large enough release history to have
gotten a significant result.

Another threat to conclusion validity of the violation density
hypothesis is that the 46\% increase in failures in {\tt all} tests
compared to {\tt jsapi} tests may not be significant given the small
sample size of tests.

\subsection{Future Work}
We can use cross-version testing for other things. For
example, library authors don't always choose the most precise version
number. They might issue a major version release that includes no
breaking changes or new features, in order to signal an increased
maturity. In this case a patch release would have been possible, and
in many ways preferable. Clients may not adopt the new version because
they are wary of breaking changes. Our cross-version testing technique
could detect that the new version still supports specification of the
previous version. As a result, clients to adopt the new version more
quickly and with less effort.


\section{Conclusion} 
We have presented a new testing discipline for writing module
interface specifications, and a cross-version testing approach to
detecting Semantic Versioning violations. We built a tool based on our
techniques, and evaluated them using several experiments. Our results
show that our techniques are effective even without interface tests
that adhere that our testing discipline.  Further, there is evidence
that our discipline is natural, in that test suites for mature
libraries may naturally evolve to align with it. Finally, adopting the
discipline does not require too much effort, and can provide even
better results.

