\section{Discussion}
One limitation of our approach is that refactoring assertions about
API secrets out of API tests can result in duplicated testing
code. This might be avoided by using a more complex test design, but
that could reducing the readability of test, making them less useful
as specifications.

We expected to find some interface tests that could not be effectively
refactored to remove API secrets. We did not find any such tests, but
the small scope of our study prevents us from making any conclusions
about the likelyhood of this scenario in practice. 

Future work: we can use cross-version testing for other things. For
example, library authors don't always choose the most precise version
number. They might issue a major version release that includes no
breaking changes or new features, in order to signal an increased
maturity. In this case a patch release would have been possible, and
in many ways preferable. Clients may not adopt the new version because
they are wary of breaking changes. Our cross-version testing technique
could detect that the new version still supports specification of the
previous version. As a result, clients to adopt the new version more
quickly and with less effort.


\section{Conculsion} 

\begin{itemize}
\item Our experiments indicate that our discipline is natural, and
  test suites for mature libraries may naturally evolve to adhere it.
\end{itemize}
